\documentclass[a4j,10pt]{jsarticle}
\usepackage{layout,url,resume}
\usepackage[dvipdfmx]{graphicx}
\pagestyle{empty}

\begin{document}
%\layout

\title{機械化された審判の導入}

% 和文著者名
\author{
    NECO B1 kosuke \thanks{}
    \and
    親:ks91 \thanks{}
}

% 和文概要
\begin{abstract}
現在のスポーツ界では、審判に人間を用いている競技が多々ある。本研究の目的は人間の審判を排除し、代わりに機械の審判を導入する事によって選手の誤審によるストレスを軽減することである。今回はテニスを具体例に、どのようなシステムをどのように導入すればよいかについて研究した。
\end{abstract}

\maketitle
\thispagestyle{empty}

\section{研究背景}

今日の日本では、さまざまなところで機械が活躍し、人間よりも正確で効率の良い仕事を行っている。これからはますます機械を用いる場所が増えていくと考えられる。しかし、スポーツの審判の世界ではいまだに人間が審判の役割を担っている。人間は機械に比べて、どうしてもミス・誤審の数・公平性に欠ける機会が多くなってしまう。誤審はプレーしている選手にとって精神面で大きなストレスとなり、選手が試合で本来の実力を発揮できなくなる原因となる可能性が非常に高い。
本研究ではテニスを具体例として、どのようなシステムをどのように導入すれば一般の人も手軽にしようでき、選手の負担を軽減することで現状よりも良い環境を提供することができるのかについて考察した。


\section{研究}

\subsection{現状のテニスのシステム}
現在テニスの公式試合は、主審と呼ばれる審判1名・先進と呼ばれる審判9名の計10人がコートを囲うように配置されボールがいんかアウトかを判定している。選手がその判定に納得できない場合には、”チャレンジシステム”によりセット内で3回のみ機械によって記録されたるリプレーを確認できる。



\subsection{ホークアイ}
先ほどのチャレンジシステムで用いられているのが、“ホークアイ”という機械である。
\\ 主に、
\begin{itemize}
\item ボールトラッキグ
\item ビデオリプレイ技術をスマートテクノロジー
\end{itemize}

を使用して、コートを囲うように設置されている複数台のカメラから、ボールとラインの関係をCGで映し出すことができる。
%---------------------------------------------

\section{実験}
\subsection{実験手法}
まずは現状のシステムに近いものに触れることが必要であると考えたため、ボールをトラッキングすることを試みた。今回はOpenCVを用いてトラッキングを試みた。
\\手法としては以下のようである
\begin{itemize}
\item フレーム間の差分画像を生成
\item 画像を2値化
\item 膨張処理して分割してしまった物体を1つの物体としてまとめる
\item 物体の重心座標(x,y)を計算し、円で囲う
\end{itemize}

\subsection{実験結果}
ボールの位置を検出することができたが、動くものに反応するため人など他のノイズも同時に検出してしまった。また、映像を用意してその解析を行うという手法をとるため、試合中にリアルタイムで表示することが難しく、また一般の人の利用も難しいを考えられる。
解像度についても十分とは言えず、より正確に映し出すことが必要であった。そのため、トラッキングシステムでは研究目的である試合を機械のみで運用すること、一般の人が手軽に使用できるという点を満たすことができなかったため改良が必要である。



\section{今後の展望}
今回はボールのトラッキングに焦点を当てて実験、研究したがこの手法ではどのように運用しても現状のシステムとあまり変わらず、リアルタイムで視聴することが難しかった。また手軽に一般の人が利用することができないため別の方法を導入する必要がある。今後は、コートをスキャンなどする事ができる機械をネットに取り付ける事によってリアルタイムで判定することができるようなものについて研究してゆく。また、この技術をどのように応用すればテニス以外のスポーツに導入でき、人間と共生できるのかについても研究することが必要である。

\section{参考文献}
“判定者について: 審判と判定テクノロジーをめぐる社会学的考察” 柏 原 全 孝 
 追手門学院大学社会学部紀要 2015年3月30日,第9号,1-15 
\\“正しい判定を作り出すテクノロジー” 柏原 全孝 
 スポーツ社会学研究 / 日本スポーツ社会学会 編, 第2号, 9-23, 1993
\\“スポーツとテクノロジー:ホークア イシステムの場合” 柏原 全孝 
 甲南女子大学研究紀要, 人間科学編(54), 145-154, 2017 
\\“大相撲のビデオ判定前史-1950 年代のテレビ中継“ (柏原全孝)





\end{document}
% end of file
